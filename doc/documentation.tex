\documentclass[11pt]{article}

\usepackage{hyperref}
\usepackage{xspace}
\usepackage{setspace}
\usepackage{paralist}
\usepackage{amsmath}

\setlength{
\topmargin}{-.5in} \setlength{
\textheight}{9in} \setlength{
\oddsidemargin}{.125in} \setlength{
\textwidth}{6.25in}

\begin{document}

\newcommand{
\sectionLabel}[2] {
\section{#1} \label{sec:#2} }
\newcommand{
\subsectionLabel}[2] {
\subsection{#1} \label{sub:#2} }
\newcommand{\fixedWidthInList}[4] {
\item[\texttt{#1} \textnormal{#2} \url{#3}] \hfill \\
#4 }
\newcommand{\optionListItem}[2] {\item[\optionFlag{#1}] #2}
\newcommand{\optionFlag}[1] {\ttt{-#1/-\MakeUppercase{#1}}}
\newcommand{\ttt}[1]{{\tt #1}\xspace}
\newcommand{\hermes}{\ttt{hermes}}
\newcommand{\multiParam}{\ttt{multi\_param}}
\newcommand{\fft}{\ttt{FFTbor2D}}
\newcommand{\rnamfpt}{\ttt{RNAmfpt}}
\newcommand{\rnaeq}{\ttt{RNAeq}}
\newcommand{\fftmfpt}{\ttt{FFTmfpt}}
\newcommand{\ffteq}{\ttt{FFTeq}}
\newcommand{\rateeq}{\ttt{RateEq}}
\newcommand{\s}{$s$\xspace}
\newcommand{\A}{$A$\xspace}
\newcommand{\B}{$B$\xspace}
\newcommand{\req}{{\em requried}}

\title{Hermes Documentation}
\author{Evan Senter}

\maketitle
\tableofcontents \clearpage

\sectionLabel{Introduction}{intro}

This documentation aims to outline the basic dependencies, installation, and usage of the Hermes RNA software suite, as well as provide extended commentary on the flags and options available to code. Particular attention is payed to the invocation style of mashup programs (using \multiParam) to dispatch flags to the appropriate submodules. To get started as quickly as possible, see the quickstart one-liner in \ref{sub:quickstart}. Should you still have questions, you can reach the main author of the source code at \href{mailto:evansenter@gmail.com}{evansenter@gmail.com}.

\sectionLabel{Installation}{install}

\subsectionLabel{Quick Start}{quickstart}

From the root directory of Hermes, execute the following command:

{\tt cd build \&\& cmake .. \&\& make}

If you encounter errors in configuring or compiling the software, we recommend checking out section \ref{sub:faq} on common troubleshooting solutions.

\subsectionLabel{Dependencies}{dependencies}
\begin{description}
  \fixedWidthInList{cmake}{($\geq$ 2.6-patch 4, tested through 3.0.0)}{http://www.cmake.org/}{ \\
  CMake is used as the build system for Hermes.} \fixedWidthInList{\textnormal{GNU99 compiler:}}{}{}{ \\
  Most C compilers should support the \ttt{-std=gnu99} flag, which is required for GNU library extensions, particularly \ttt{unistd.h}} \fixedWidthInList{\textnormal{C++11 compiler:}}{}{}{ \\
  C++11 support (included in \ttt{g++} $\geq$ 4.7) is necessary for proper struct initialization in \fft.} \fixedWidthInList{\textnormal{OpenMP support}}{}{http://openmp.org/wp/}{ \\
  OpenMP support comes by default in most modern compilers, and is required for loop-optimization in \fft.} \fixedWidthInList{LAPACK}{($\geq$ 3.4.2)}{http://www.netlib.org/lapack/}{ \\
  Various LAPACK routines are used in \rnamfpt to compute the inverse, or pseudoinverse, of a transition probability matrix.} \fixedWidthInList{GSL}{($\geq$ 1.15, tested through 1.16)}{http://www.gnu.org/software/gsl/}{ \\
  GSL is required to compute the eigendecomposition of a (possibly) non-symmetric transition rate matrix for \rnaeq.} \fixedWidthInList{FFTW3}{($\geq$ 3.3.4)}{http://www.fftw.org/}{ \\
  FFTW3 functions are used to compute the inverse discrete Fourier transform in \fft.} \fixedWidthInList{libRNA.a}{($\geq$ 2.0.7, tested through 2.1.7)}{http://www.tbi.univie.ac.at/RNA/}{ \\
  Various ViennaRNA functions and data structures are leveraged for homogenous energy model support, as well as \ttt{fold\_par}, \ttt{pf\_fold\_par}, and \ttt{subopt\_par}. We additionally make use of ViennaRNA functions to compute the necessary polynomial size for \fft, determined in the following fashion. Parameters $K$ (resp. $L$) are defined to be the sum of the number of base pairs in reference structure $A$ (resp. reference structure $B$) plus the number of base pairs in the maximum matching (Nussinov) structure which contains no base pair of $A$ (resp. $B$).}
\end{description}

\subsectionLabel{Compilation}{compile}
\begin{description}
  \item[{\tt cd build \&\& cmake ..}] \hfill \\
  First, ensure that your system has the dependencies outlined above (the presence of CMake can be verified with \ttt{cmake --version}). While not required, is widely considered {\em best practice} to perform an out-of-source build, where the compilation of the code happens in a separate directory from the location of the source itself. To this end, \hermes provides an empty \ttt{build} directory that can be used for compiling the code. From \ttt{hermes/build} execute \ttt{cmake} with the path to \ttt{hermes/CMakeLists.txt} (\ttt{..}) provided as argument.
  \item[{\tt make}] \hfill \\
  Compiles the code, and generates binary executables for \fft, \rnamfpt, \rnaeq, \fftmfpt, \ffteq, and \rateeq. Additionally generates both static and shared libraries for \fft, \rnamfpt, and \rnaeq. The output directory for binaries is \ttt{hermes/bin} and output directory for libraries is \ttt{hermes/lib}.
  \item[\ttt{make install} \textnormal{(optional)}] \hfill \\
  Installs the executables built with \ttt{make} to \ttt{\$DESTDIR/bin} and copies libraries / archives to \ttt{\$DESTDIR/lib} (on *nix systems, \ttt{\$DESTDIR} defaults to \ttt{/usr/local}).
\end{description}

\subsectionLabel{Troubleshooting}{faq}

While we have done the utmost to try and ensure that CMake is able to infer locations of third-party libraries and add compiler-appropriate flags in an automated fashion, due to the diversity of build environments possible, it is possible that you will need to specify additional command-line flags to \ttt{cmake} when generating the Makefiles in order to successfully build Hermes. The following are five useful flags for CMake, and a brief explanation of when they may need to be employed:
\begin{itemize}
  \item The default compiler I'd like to use for C code is installed in a non-standard location, or not the globally default C compiler.
  \begin{description}
    \item[CMAKE\_C\_COMPILER] \hfill \\
    i.e. {\tt cmake -DCMAKE\_C\_COMPILER=/path/to/c/compiler ..} \hfill \\
    \begin{spacing}
      {.5}
    \end{spacing}
    This variable sets the path to the compiler to use for configuration and subsequent compilation via \ttt{make}. This is the compiler that will be used by CMake to test for the presence of various flags, i.e. \ttt{-O3} and \ttt{-Wall}.
  \end{description}

  \item The default compiler I'd like to use for C++ code is installed in a non-standard location, or not the globally default C++ compiler.
  \begin{description}
    \item[CMAKE\_CXX\_COMPILER] \hfill \\
    i.e. {\tt cmake -DCMAKE\_CXX\_COMPILER=/path/to/cxx/compiler ..} \hfill \\
    \begin{spacing}
      {.5}
    \end{spacing}
    Same as above.
  \end{description}

  \item I'm getting a compile-time error indicating undefined symbols for \ttt{\_get\_iindx}, \\
  \ttt{\_maximumMatchingConstraint} or something similar.
  \begin{description}
    \item[CMAKE\_LIBRARY\_PATH] \hfill \\
    i.e. {\tt cmake -DCMAKE\_LIBRARY\_PATH=/dir/for/libRNA-2.0.7+/ ..} \hfill \\
    \begin{spacing}
      {.5}
    \end{spacing}
    These are \ttt{libRNA.a} symbols specific to the $2.0+$ release of ViennaRNA. In all cases identified thus far, this error means that the version of \ttt{libRNA.a} found by CMake is not out of date, and can be resolved by explicitly providing the library path to a $2.0+$ ViennaRNA static library using the \ttt{CMAKE\_LIBRARY\_PATH} flag.
  \end{description}

  \item Libraries required by Hermes are not installed in a location visible by the linker (in LD\_LIBRARY\_PATH), and CMake is unable to validate their existence.
  \begin{description}
    \item[CMAKE\_LIBRARY\_PATH] \hfill \\
    i.e. {\tt cmake -DCMAKE\_LIBRARY\_PATH="/more/libraries;/even/more/libraries" ..} \hfill \\
    In the case when libraries required by Hermes are not visible to CMake, or the global library is an out-of-date version, it is possible to provide hints to the build tool for additional directories to search. Directories specified by the CMAKE\_LIBRARY\_PATH flag will be prepended onto the linker search path, and thus override global matches (handling the case where default libraries aren't sufficiently up to date). When desiring to provide multiple locations to search for libraries, CMake uses the semicolon (;) character as a separator and the entire string should be quoted to escape the shell environment.
  \end{description}

  \item Headers required by Hermes are not installed in a location visible by the compiler (in CPATH or a derivative), and as a result I'm seeing \ttt{undefined reference to} errors.
  \begin{description}
    \item[CMAKE\_INCLUDE\_PATH] \hfill \\
    i.e. {\tt cmake -DCMAKE\_INCLUDE\_PATH="/more/includes;/even/more/includes" ..} \hfill \\
    \begin{spacing}
      {.5}
    \end{spacing}
    It is generally likely that the CMAKE\_LIBRARY\_PATH and CMAKE\_INCLUDE\_PATH will both be necessary, when either one is required. This flag operates in a fashion identical to CMAKE\_LIBRARY\_PATH described above, and uses the same syntax. Alternatively a user can update their \ttt{CPATH} environment variable, but this may have unpredictable results when headers are found, but out of date.
  \end{description}

  \item I don't have permissions to \ttt{make install} to the default location (generally \ttt{/usr/local} for *nix) on my system.
  \begin{description}
    \item[CMAKE\_INSTALL\_PREFIX] \hfill \\
    i.e. {\tt cmake -DCMAKE\_INSTALL\_PREFIX=/make/install/path/prefix ..} \hfill \\
    \begin{spacing}
      {.5}
    \end{spacing}
    This variable sets the destination directory of the \ttt{make install} command. Binaries will be placed in the \ttt{bin} subdirectory and libraries will be placed in the \ttt{lib} subdirectory. This is analogous to \ttt{./configure --prefix=/make/install/path/prefix} in Autotools and can also be achieved by setting the \ttt{DESTDIR} environment variable.
  \end{description}
\end{itemize}

\sectionLabel{Software Organization}{organization}

\subsectionLabel{General Principles}{principles}

The Hermes code is organized into two major directories, \ttt{hermes/src} and \ttt{hermes/mashup}. The conceptual difference between these two directories is that \ttt{hermes/src} code (\fft, \rnamfpt, \rnaeq) are all stand-alone pieces of software which achieve specific goals (computing energy grids, hitting time, and equilibrium time respectively). Alternatively, code residing in \ttt{hermes/mashup} aims to leverage general concepts across the Hermes package to provide a means to ask even more specific questions. In example, \fft allows an investigator to compute the 2D energy grid correspondent to an input RNA sequence $s$ and two structures $A, B$. \rnaeq can estimate the population occupancy of $A, B$ for $s$ by either sampling suboptimal structures or exhaustive structural enumeration, but these approaches aren't tractable for non-trivial RNAs.

\ffteq, located in \ttt{hermes/mashup/population\_from\_fftbor2d} uses functions from \\
\ttt{libfftbor\_static.a} (derived from \fft) to compute the energy landscape and functions from \ttt{librnaeq\_static.a} (derived from \rnaeq) to estimate population occupancy for $s, A, B$ without requiring an investigator to copy pieces of individual packages to achieve their goals, instead using the static libraries automatically produced for all \ttt{hermes/src} software, shared headers made available in \ttt{hermes/h} and \ttt{libmulti\_param.a} (from \ttt{hermes/src/multi\_param}) to dispatch command-line arguments to the appropriate underlying function. The result? The entirety of \ffteq is 67 lines of C++ code, and uses native binary data structures the entire way through; there is no command-line funneling of \fft into \rnaeq.

\subsectionLabel{multi\_param Overview}{mpoverview}

All of \fft, \rnamfpt, and \rnaeq present a wide selection of command-line arguments to meet the diverse demands of end users. When we moved on to developing {\em mashup} software between these three programs, there were a number of requirements that we came up with to make both implementing and using these programs as easy as possible. We decided that
\begin{inparaenum}
  [\itshape a\upshape)]
  \item the developer should not have to reimplement command-line parsing for mashups
  \item all existing flags for underlying libraries (i.e. \fft, \rnamfpt, \rnaeq) would be supported, and
  \item flags would be namespaced to have deterministic targets.
\end{inparaenum}

To achieve these goals, the library \multiParam was developed. This library simply takes a collection of command line arguments and re-dispatches them to the appropriate underlying library. Given a set of command-line flags such as \ttt{--all-v --fftbor2d-i GGGAAACCC --fftbor2d-j `.........' --fftbor2d-k `(((...)))' --mfpt-x --mfpt-h}, the code ensures that \fft is passed \ttt{-v -i GGGAAACCC -j `.........' -k `(((...)))'} as options in \ttt{argv} and \rnamfpt is passed \ttt{-v -x -h} as options in \ttt{argv}.

\subsectionLabel{multi\_param Details}{mpdetails}

An example from \ttt{hermes/mashup/mfpt\_from\_fftbor2d/mfpt\_from\_fftbor2d.cpp}:
\begin{verbatim}
PARAM_CONTAINER* params;
FFTBOR2D_PARAMS fftbor2d_params;

/* ...omitted for clarity... */

char* subparams[] = { "fftbor2d", "mfpt" };
params = split_args(argc, argv, subparams, 2);

fftbor2d_params = init_fftbor2d_params();
parse_fftbor2d_args(
  fftbor2d_params,
  params[0].argc,
  params[0].argv,
  &mfpt_from_fftbor2d_usage
);
\end{verbatim}

The way this code works is by first declaring which packages can be used by the mashups, out of \ttt{fftbor2d}, \ttt{mfpt}, or \ttt{population}. In the snippet above the selection is saved in the variable \ttt{subparams}. All of the \fft, \rnamfpt, and \rnaeq libraries have \ttt{init\_*\_params} functions available, which return an object with the default parameters for that package. They also all have \ttt{parse\_*\_args} functions, which take three arguments,
\begin{inparaenum}
  [\itshape 1\upshape)]
  \item a pointer to the parameters object
  \item \ttt{argc}, and
  \item \ttt{argv}
\end{inparaenum}
.

\ttt{split\_args} takes the prefixed command-line arguments (see the example in \ref{sub:mpoverview}) and bins them by their prefix, with the special \ttt{--all} prefix being supplied to all declared subparams. The prefixes are then removed and the grouped arguments are returned from the function in a \ttt{PARAM\_CONTAINER} array, with the same order as the subparams were passed to the function. It is then trivial to call the \ttt{parse\_*\_args} functions with the corresponding subarrays from \ttt{PARAM\_CONTAINER} to get a final parameters object.

Leveraging the example from \ref{sub:mpoverview} a final time, the variable \ttt{params} would look as follows after invoking \ttt{split\_args}:
\begin{verbatim}
[
  {
    argv: ["-v", "-i", "GGGAAACCC", "-j", ".........", "-k", "(((...)))"],
    argc: 7
  }, {
    argv: ["-v", "-x", "-h"],
    argc: 3
  }
]
\end{verbatim}

\sectionLabel{Core Programs}{coreprogs}

\subsectionLabel{{\tt FFTbor2D}}{fft}

\subsubsection{Applications} \label{subsub:fftborapp}

\fft computes the 2D energy landscape of an arbitrary but fixed sequence \s, parameterized by base pair distance from input structures \A and \B.

\subsubsection{Example Usage} \label{subsub:fftborex}

\begin{verbatim}
FFTbor2D -m -i GGGAAACCC -j '.........' -k '(((...)))' -e ./misc/rna_turner2004.par

+0.00000000  +0.00000000  +0.00000000  +0.10047371  +0.00000000  +0.00000000  +0.000000000
+0.00000000  +0.00000000  +0.00019384  +0.00000000  +0.00181492  +0.00000000  +0.000000000
+0.00000000  +0.02666435  +0.00000000  +0.00036362  +0.00000000  +0.16642614  +0.000000000
+0.70406341  +0.00000000  +0.00000000  +0.00000000  +0.00000000  +0.00000000  +0.000000000
+0.00000000  +0.00000000  +0.00000000  +0.00000000  +0.00000000  +0.00000000  +0.000000000
+0.00000000  +0.00000000  +0.00000000  +0.00000000  +0.00000000  +0.00000000  +0.000000000
+0.00000000  +0.00000000  +0.00000000  +0.00000000  +0.00000000  +0.00000000  +0.000000000
\end{verbatim}

In the above example, position $(0, 0)$ is located in the upper-left of the output matrix, and base pair distance from \A (resp. \B) moves along the rows (resp. columns). This can be verified using the \optionFlag{c} flag for CSV output instead of matrix formatting (using \optionFlag{m}):

\begin{verbatim}
FFTbor2D -c -i GGGAAACCC -j '.........' -k '(((...)))' -e ./misc/rna_turner2004.par

0,3,+0.10047371
1,2,+0.00019384
1,4,+0.00181492
2,1,+0.02666435
2,3,+0.00036362
2,5,+0.16642614
3,0,+0.70406341
\end{verbatim}

\subsubsection{Options} \label{subsub:fftboropt}

\begin{itemize}
  \optionListItem{i}{\req. The RNA sequence \s to be used by \fft.}
  \optionListItem{j}{\req. The first structure \A to be used by \fft. Any base pairs in \A incompatible with \s (i.e. not a Watson-Crick or GU wobble) are ignored by \fft.}
  \optionListItem{k}{\req. The second structure \B to be used by \fft. The same restrictions apply as with \optionFlag{j}.}
  \optionListItem{t}{The temperature at which the energy landscape is computed. The provided value is used both in computing Boltzmann factors ($\beta = -1/RT$) and for energy table lookups. The default is $37^{\circ}$C.}
  \optionListItem{e}{The energy file to be used by \fft, in the Vienna 2.0 format. The default is \ttt{rna\_turner2004.par} located in the same directory as the \fft executable.}
  \optionListItem{p}{The precision $m$ of the probabilities output by \fft, in base 2 format. The default is system specific (you can call \fft with no parameters to see the precision range available on your platform), and using $0$ disables any precision control---not recommended. The default is 27, which corresponds to 8 decimal places of precision.}
  \optionListItem{b}{Enables the output of performance related benchmarking for all major subroutines in \fft.}
  \optionListItem{v}{Enables verbose output.}
  \item[Output formatting flags (mutually exclusive):] \hfill
    \begin{itemize}
      \item[] The default format is a verbose version of \optionFlag{s} which also includes the user-input \s, \A, \B, and column headers.
      \optionListItem{c}{CSV-formatted output, where only non-zero positions are emitted. The first (resp. second) column corresponds to base pair distance from \A (resp. \B), and the final column is the Boltzmann probability ($p(Z_{k, l}/Z)$) for that position.}
      \optionListItem{m}{Output formatted as a matrix, where $(0, 0)$ is in the lop-left and base pair distance from \A (resp. \B) moves along the rows (resp. columns).}
      \optionListItem{s}{Output formatted in the same fashion as \optionFlag{c}, but delimited with tab characters. The rightmost column is the ensemble free energy ($-RT\log(Z_{k, l})$) in \footnotesize$\dfrac{kcal}{mol}$.}
    \end{itemize}
\end{itemize}

\subsectionLabel{{\tt RNAmfpt}}{rnamfpt}

\subsubsection{Applications} \label{subsub:rnamfptapp}

\rnamfpt computes the mean first passage time (hitting time) of an user-provided input matrix, in CSV format. The user can provide as input either a 2D energy grid (such as those produced by \fft with the \optionFlag{c} format) or a transition probability matrix. The default expectation is a 2D energy grid composed of Boltzmann probabilities, though ensemble free energies are alternatively supported with \optionFlag{e}.

If providing a transition probability matrix $M$ as input (with the \optionFlag {t} flag), the format is still expected to be in CSV form, where columns $i$, $j$, $p$ represent the 0-indexed row-based probabilities for $M$, s.t. $M_{i, j} = p_{i\to j} = (i, j,p_{i\to j})$. Only non-zero $M_{i, j}$ entries are required.

\subsubsection{Example Usage} \label{subsub:rnamfptex}

These trivial examples make use of the file \ttt{./src/mfpt/example.csv}, which is the output of calling:

\begin{verbatim}
FFTbor2D -c -i GGGAAACCC -j '.........' -k '(((...)))' -e ./misc/rna_turner1999.par

0,3,+0.10531278
1,2,+0.00213850
1,4,+0.00509026
2,1,+0.28092942
2,3,+0.00032964
2,5,+0.15262286
3,0,+0.45357654
\end{verbatim}

\noindent A simple invocation of \rnamfpt appears like:

\begin{verbatim}
RNAmfpt -c ./src/mfpt/example.csv -xh

+699.65148561
\end{verbatim}

In the above example, the value output by \rnamfpt is the approximate hitting time for the 2D energy grid provided with the \optionFlag{c} flag. \rnamfpt converts this input to a transition probability matrix where only single base pair substitutions are valid (diagonal moves only using \optionFlag{x}), and the Hastings correction is applied with \optionFlag{h}.

\subsubsection{Options} \label{subsub:rnamfptopt}

\begin{itemize}
  \optionListItem{c}{\req. Path to the input CSV file.}
  \item[Input format flags (mutually exclusive):] \hfill
    \begin{itemize}
      \item[] The default expectation is a 0-indexed 2D probability landscape in the CSV format described within \ref{subsub:rnamfptapp}. In this case, transition probability $p_{a\to b} = \min(1, \exp(-(p_b / p_a) / RT)) / N_a$, where $p_a$, $p_b$ are Boltzmann probabilities and $N_a$ is the number of neighbors of $a$.
      \optionListItem{e}{The input matrix is comprised of ensemble free energies rather than Boltzmann probabilities. If this flag is provided, the transition probability $p_{a\to b} = \min(1, \exp(-(E_b - E_a) / RT)) / N_a$ where definitions follow as above.}
      \optionListItem{t}{The input CSV file is a transition probability matrix already, and \rnamfpt should not try to convert it into one. In this case, the first two columns in the CSV file are 0-indexed row-order indices into the transition probability matrix, and the third (final) column is the transition probability $p_{a\to b}$.}
    \end{itemize}
  \item[Transition probability matrix manipulation:] \hfill
    \begin{itemize}
      \optionListItem{x}{Only permit single base pair moves: $(i, j) \to (k, l) \implies (k, l) \in (i\pm 1, j\pm 1)$. This option assumes that the input is not already a transition probability matrix, the matrix is parameterized by base pair distance to some fixed, implicit \A, \B and that the matrix already satisfies the triangle inequality and parity condition.}
      \optionListItem{f}{The transition probability matrix generated by \rnamfpt should be fully connected.}
      \optionListItem{t}{The input CSV file is a transition probability matrix already, and \rnamfpt should not try to convert it into one. In this case, the first two columns in the CSV file are 0-indexed row-order indices into the transition probability matrix, and the third (final) column is the transition probability $p_{a\to b}$.}
      \optionListItem{h}{Enables the usage of the Hastings adjustment in formulating the transition probability matrix. Has no effect if the input is already a transition matrix (using \optionFlag{t}) or fully connected (using using \optionFlag{f}). When enabled, the transition probabilities are defined as $p_{a\to b} = \min(1, (N_a / N_b) \times \exp(-(p_b / p_a) / RT)) / N_a$, where variable definitions are the same as described above. Computation of $N_a$, $N_b$ respects grid boundaries and the triangle inequality, with base pair distance $d_{bp}(A,B)$ inferred from the input energy landscape by looking for a non-zero $(0, d_{bp}(A,B))$ or $(d_{bp}(A,B), 0)$ position. Also works with the \optionFlag{e} flag to use energies instead of probabilities.}
    \end{itemize}
  \item[Using $\epsilon$ to inflate the energy landscape:] \hfill
    \begin{itemize}
      \item[] In instances where no direct path exists from \A to \B in single base pair steps, or one would like to adjust the energy landscape such that all valid positions are accessible, we provide a number of flags to make an $\epsilon$ adjustment to the input 2D probability landscape. Valid positions are defined as $V = \{ (i, j) \mid 0 \leq i, j \leq n \wedge i + j \geq d_{bp}(A, B) \wedge i + j \equiv d_{bp}(A, B) \;(\bmod\; 2) \}$, where $n$ is set with \optionFlag{n} and $d_{bp}(A, B)$ with \optionFlag{d} (if it can't be inferred). All probabilities $\{ p_{i, j} \mid (i, j) \in V \}$ are then $\epsilon$-adjusted such that $p^*_{i, j} = \frac{p_{i, j} + \frac{\epsilon}{\left\vert{V}\right\vert}}{1 + \epsilon}$.
      \optionListItem{d}{Flag to explicitly provide $d_{bp}(A,B)$, as defined in \optionFlag{h}. Only required when $d_{bp}(A,B)$ can't be inferred from the input probability matrix by identifying a non-zero $(0, d_{bp}(A,B))$ or $(d_{bp}(A,B), 0)$ position in the input.}
      \optionListItem{n}{Maximum base pair distance $n$ to use when including all accessible positions.}
      \optionListItem{o}{The {\em total} $\epsilon$ probability to add to the input probability grid before renormalization. The {\em per-position} contribution will be $\frac{\epsilon}{\left\vert{V}\right\vert}$, where definitions are as above.}
    \end{itemize}
  \item[Mean first passage time options:] \hfill
    \begin{itemize}
      \optionListItem{a}{The 0-indexed position of $(0, d_{bp}(A,B))$ in the input CSV file, representing the starting state for folding. If this flag is not provided, \rnamfpt will attempt to infer it by looking for a non-zero $(0, d_{bp}(A,B))$ position in the input and present an error to the user if it can't be found. If the input is already a transition matrix (using \optionFlag{t}), \optionFlag{a} should be the index of the row / column correspondent to the start state.}
      \optionListItem{z}{The 0-indexed position of $(d_{bp}(A,B), 0)$ in the input CSV file, representing the target state for folding. Expectations are identical as with \optionFlag{a}.}
      \optionListItem{r}{Compute a transition rate matrix rather than a probability matrix. The rate matrix $M$ is defined such that $q_{i,i} = -\sum\limits_{j\neq i}q_{i,j}$. This flag is provided so that \rnaeq and extensions involving rate matrices can use a unified codebase for their generation, and is not intended for typical calls to \rnamfpt.}
      \optionListItem{p}{Use the Moore-Penrose pseudoinverse to invert the transition probability matrix. This can be useful when an approximate solution is desired for input which produces a singular transition probability matrix.}
      \optionListItem{l}{Prints all mean first passage times $\{\text{mfpt}_{X\to B} \mid X \in V \wedge X \neq B\}$ where $V$ is the set of valid positions for the given input.}
    \end{itemize}
  \optionListItem{v}{Enables verbose output.}
\end{itemize}

\subsectionLabel{{\tt RNAeq}}{rnaeq}

  % fprintf(stderr, "-A/a\tstart state,           default is inferred. If provided, should indicate the 0-indexed position in the transition matrix corresponding to the starting structure (see options for -k).\n");
  % fprintf(stderr, "-B/b\t(b)enchmarking,        default is disabled. When enabled, RNAeq will print benchmarking times for internal function calls.\n");
  % fprintf(stderr, "-C/c\tenergy (c)ap,          default is disabled. When provided, RNAsubopt will only sample structures within energy_cap kcal/mol of the MFE structure.\n");
  % fprintf(stderr, "-D/d\t(d)elta range,         default is disabled. When provided, this value specifies the delta-size required for the population to be approaching equilibrium. This position is used as a starting point for a more fine-grained scan using the -e and -w values.\n");
  % fprintf(stderr, "-E/e\t(e)psilon,             default is disabled. When provided, we will output the time at which the equilibrium stable within -e for a window of size -w.\n");
  % fprintf(stderr, "-F/f\tbin (f)ilename,        Provided in conjunction with the -r flag to specify the read / write file for serializing the eigensystem.\n");
  % fprintf(stderr, "-G/g\tei(g)envalues only,    default is disabled. When enabled, RNAeq will only generate the eigenvalues for the transition rate matrix.\n");
  % fprintf(stderr, "-H/h\tall subpop. in eq.,    default is disabled. When provided alongside the -q flag, the equilibrium time is estimated for when *all* subpopulations approach equilibrium, rather than when just the start-to-end state curve (-a to -z) doesn't deviate more than epsilon (-e) within the window size (-w).\n");
  % fprintf(stderr, "-I/i\tstart time,            natrual log of the starting time for computing population proportion.\n");
  % fprintf(stderr, "-J/j\tend time,              natrual log of the ending time for computing population proportion.\n");
  % fprintf(stderr, "-K/k\tstarting structure,    structure for which the probability at time 0 is equal to 1. If not provided, the empty structure is used / inferred.\n");
  % fprintf(stderr, "-L/l\tending structure,      structure of interest for detailed population proportion values. If not provided, the MFE structure is used / inferred.\n");
  % fprintf(stderr, "-N/n\t(n)o soft bounds,      default is disabled. When enabled, the population proportion / equilibrium time will not use soft bounds computed with the -d delta value, and instead will compute population proportion / equilibrium over the entire timespan (timespan can be adjusted with -i and -j).\n");
  % fprintf(stderr, "-O/o\tl(o)nely basepairs,    default is disabled. When enabled, RNAsubopt will sample structures containing lonely basepairs.\n");
  % fprintf(stderr, "-P/p\tste(p) size,           natrual log of the step size for computing population proportion (start_time < start_time + step_size <= end_time).\n");
  % fprintf(stderr, "-Q/q\te(q)uilibrium,         Compute the equilibrium time for the target state. By default uses an exact method that requires a target structure. If the -e flag is passed as well, uses an aproximation technique suitable for any probability rate matrix.\n");
  % fprintf(stderr, "-R/r\tse(r)ialize direction, default is disabled. If passed with the value '1', then the eigensystem and its inversion will be serialized to the file specified by the -f flag. If passed with the value '-1', the file specified by the -f flag will be deserialized and used for computing the data of interest.\n");
  % fprintf(stderr, "-S/s\t(s)equence,            sequence of interest for computing population proportions.\n");
  % fprintf(stderr, "-T/t\t(t)emperature,         temperature at which suboptimal structures are generated. This value is passed to (and only used by) ViennaRNA's RNAsubopt.\n");
  % fprintf(stderr, "-V/v\t(v)erbose,       default is disabled, presents some debug information at runtime.\n\n");
  % fprintf(stderr, "-W/w\t(w)indow size,         default is 5. Specifies the window size (exclusive) for predicting equilibrium. Equilibrium is considered as having been achieved when all indicies (i + 1)..(i + window_size - 1) are within epsilon (-e) of the population proportion at time i.\n");
  % fprintf(stderr, "-Z/z\tend state,             default is inferred. If provided, should indicate the 0-indexed position in the transition matrix corresponding to the ending structure (see options for -l).\n\n");


\sectionLabel{Mashups}{mashupprogs}

\subsectionLabel{{\tt FFTmfpt}}{fftmfpt}

\subsectionLabel{{\tt FFTeq}}{ffteq}

\subsectionLabel{{\tt RateEq}}{rateeq}

\sectionLabel{References}{ref}

\end{document}
